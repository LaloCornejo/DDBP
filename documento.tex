\documentclass[12pt,letterpaper]{article}

% Paquetes básicos
\usepackage[utf8]{inputenc}
\usepackage[spanish]{babel}
\usepackage{geometry}
\usepackage{graphicx}
\usepackage{hyperref}
\usepackage{listings}
\usepackage{xcolor}
\usepackage{float}
\usepackage{enumitem}
\usepackage{enumitem}
\usepackage{titlesec}
\usepackage{fancyhdr}
\usepackage{tikz}
\usetikzlibrary{positioning,arrows,shapes}
\geometry{
    a4paper,
    total={170mm,257mm},
    left=20mm,
    right=20mm,
    top=20mm,
    bottom=20mm,
}

% Configuración de encabezados y pies de página
\pagestyle{fancy}
\fancyhf{}
\fancyhead[L]{Mongo Red Social Distribuida}
\fancyhead[R]{\thepage}
\fancyfoot[C]{DDBP - Documentación}

% Configuración para listados de código
\definecolor{codebackground}{rgb}{0.95,0.95,0.95}
\definecolor{codestring}{rgb}{0.2,0.6,0.2}
\definecolor{codekeyword}{rgb}{0.7,0.1,0.1}
\definecolor{codecomment}{rgb}{0.5,0.5,0.5}

\lstset{
    backgroundcolor=\color{codebackground},
    commentstyle=\color{codecomment},
    keywordstyle=\color{codekeyword}\bfseries,
    stringstyle=\color{codestring},
    basicstyle=\ttfamily\small,
    breakatwhitespace=false,
    breaklines=true,
    captionpos=b,
    keepspaces=true,
    showspaces=false,
    showstringspaces=false,
    showtabs=false,
    tabsize=2
}

% Título, autor y fecha
\title{\LARGE \textbf{Red Social Distribuida - MongoDB}}

\author{
    Jesus Eduardo Cornejo Clavel \\
    \textit{ALUMNO} 
    \and
    Eduardo Cornejo-Velazquez \\
    \textit{PROFESOR} 
}

% \date{\today}

\begin{document}

\maketitle
\tableofcontents
\newpage

% 1. Arquitectura del Sistema
\section{Arquitectura del Sistema}
\subsection{Configuración de MongoDB}
\subsubsection{Configuración del Replica Set}
La aplicación implementa una arquitectura de alta disponibilidad basada en MongoDB con una configuración de Replica Set. Esta configuración consta de tres nodos:

\begin{itemize}
    \item \textbf{Nodo Primario (Primary)}: Gestiona todas las operaciones de escritura y coordina la sincronización con los nodos secundarios.
    \item \textbf{Nodos Secundarios (Secondary)}: Dos nodos que mantienen copias sincronizadas de los datos y pueden asumir el rol primario en caso de fallos.
\end{itemize}

Esta arquitectura proporciona:
\begin{itemize}
    \item \textbf{Alta disponibilidad}: Si el nodo primario falla, uno de los secundarios puede ser promovido automáticamente.
    \item \textbf{Redundancia de datos}: Los datos se replican en múltiples nodos, evitando pérdidas en caso de fallos.
    \item \textbf{Balanceo de lecturas}: Las operaciones de lectura pueden distribuirse entre los nodos secundarios.
\end{itemize}

La configuración se implementa mediante Podman Compose:

\begin{lstlisting}[language=yaml]
services:
  # Nodo MongoDB primario
  central-mongodb:
    image: mongo:latest
    container_name: central-mongodb
    command: mongod --replSet rs0 --keyFile /etc/mongo-keyfile --bind_ip_all --auth --port 27017
    ports:
      - "27017:27017"
    volumes:
      - ./mongo-keyfile:/etc/mongo-keyfile:ro
    environment:
      - MONGO_INITDB_ROOT_USERNAME=admin
      - MONGO_INITDB_ROOT_PASSWORD=password
    networks:
      - ddbp_mongo-network

  # Primer nodo MongoDB secundario
  secondary-mongodb-1:
    image: mongo:latest
    container_name: secondary-mongodb-1
    command: mongod --replSet rs0 --keyFile /etc/mongo-keyfile --bind_ip_all --auth --port 27017
    ports:
      - "27018:27017"
    volumes:
      - ./mongo-keyfile:/etc/mongo-keyfile:ro
    environment:
      - MONGO_INITDB_ROOT_USERNAME=admin
      - MONGO_INITDB_ROOT_PASSWORD=password
    networks:
      - ddbp_mongo-network
    depends_on:
      - central-mongodb

  # Segundo nodo MongoDB secundario
  secondary-mongodb-2:
    image: mongo:latest
    container_name: secondary-mongodb-2
    command: mongod --replSet rs0 --keyFile /etc/mongo-keyfile --bind_ip_all --auth --port 27017
    ports:
      - "27019:27017"
    volumes:
      - ./mongo-keyfile:/etc/mongo-keyfile:ro
    environment:
      - MONGO_INITDB_ROOT_USERNAME=admin
      - MONGO_INITDB_ROOT_PASSWORD=password
    networks:
      - ddbp_mongo-network
    depends_on:
      - central-mongodb
\end{lstlisting}
\textit{(Ruta: /docker-compose.yml - Configuración de contenedores MongoDB para ReplicaSet)}

El proceso de inicialización del Replica Set se maneja mediante un contenedor adicional que ejecuta un script para configurar los nodos como un conjunto replicado:

\begin{lstlisting}[language=yaml]
  # Contenedor de configuración de MongoDB - inicializa el replica set
  mongo-setup:
    image: mongo:latest
    container_name: mongo-setup
    restart: "no"
    depends_on:
      - central-mongodb
      - secondary-mongodb-1
      - secondary-mongodb-2
    networks:
      - ddbp_mongo-network
    volumes:
      - ./setup-replica.sh:/setup-replica.sh:ro
    entrypoint: ["/bin/bash", "/setup-replica.sh"]
\end{lstlisting}
\textit{(Ruta: /docker-compose.yml - Configuración de contenedor de inicialización para MongoDB)}

\begin{figure}[H]
    \centering
    \begin{tikzpicture}[node distance=2cm]
        % MongoDB Container Architecture
        \node[rectangle,draw,minimum width=3cm,minimum height=1.5cm] (setup) {Mongo Setup};
        \node[rectangle,draw,minimum width=3cm,minimum height=1.5cm,below=1cm of setup] (primary) {Primary Node};
        \node[rectangle,draw,minimum width=3cm,minimum height=1.5cm,below left=1cm of primary] (secondary1) {Secondary 1};
        \node[rectangle,draw,minimum width=3cm,minimum height=1.5cm,below right=1cm of primary] (secondary2) {Secondary 2};
        
        % Network container
        \node[rectangle,draw,dashed,fit=(setup) (primary) (secondary1) (secondary2),inner sep=0.5cm] (network) {};
        \node[above] at (network.north) {ddbp\_mongo-network};
        
        % Connections
        \draw[->,thick] (setup) -- (primary);
        \draw[->,thick] (setup) -- (secondary1);
        \draw[->,thick] (setup) -- (secondary2);
        \draw[->,thick] (primary) -- (secondary1);
        \draw[->,thick] (primary) -- (secondary2);
    \end{tikzpicture}
    \caption{Arquitectura de contenedores MongoDB con Replica Set}
    \label{fig:mongodb-containers}
\end{figure}

\begin{figure}[H]
    \centering
    \begin{tikzpicture}[node distance=2cm]
        % Central MongoDB node
        \node[rectangle,draw,minimum width=3cm,minimum height=2cm] (primary) {Primary Node\\(Priority 10)};
        
        % Secondary nodes
        \node[rectangle,draw,minimum width=3cm,minimum height=2cm,below left=2cm of primary] (secondary1) {Secondary 1\\(Priority 5)};
        \node[rectangle,draw,minimum width=3cm,minimum height=2cm,below right=2cm of primary] (secondary2) {Secondary 2\\(Priority 1)};
        
        % Application node
        \node[rectangle,draw,minimum width=3cm,minimum height=1cm,above=2cm of primary] (app) {Rust Application};
        
        % Arrows for replication
        \draw[->,thick] (primary) -- (secondary1) node[midway,left] {Replication};
        \draw[->,thick] (primary) -- (secondary2) node[midway,right] {Replication};
        
        % Arrows for application communication
        \draw[->,thick] (app) -- (primary) node[midway,right] {Writes};
        \draw[->,thick,dashed] (app) -- (secondary1) node[midway,left] {Reads};
        \draw[->,thick,dashed] (app) -- (secondary2) node[midway,right] {Reads};
    \end{tikzpicture}
    \caption{Arquitectura del ReplicaSet MongoDB con esquema de prioridad y flujo de datos}
    \label{fig:replicaset-architecture}
\end{figure}

\subsubsection{Arquitectura de ReplicaSet y Priorización}
La implementación del ReplicaSet en nuestra arquitectura sigue un esquema de priorización estratégica (10-5-1) que garantiza un comportamiento predecible durante las elecciones de nodo primario y la distribución de cargas de trabajo:

\begin{itemize}
    \item \textbf{Nodo Central (Prioridad 10)}: Configurado como el servidor preferido para el rol primario, este nodo tiene la prioridad más alta para garantizar estabilidad en condiciones normales de operación.
    \item \textbf{Primer Nodo Secundario (Prioridad 5)}: Mantiene una prioridad intermedia, sirviendo como primer candidato de respaldo para asumir el rol primario en caso de fallo del nodo central.
    \item \textbf{Segundo Nodo Secundario (Prioridad 1)}: Con la prioridad más baja, este nodo está diseñado principalmente para mantener redundancia y servir lecturas, asumiendo el rol primario solo en situaciones extremas.
\end{itemize}

La configuración del ReplicaSet se implementa mediante un script de inicialización automatizado:

\begin{lstlisting}[language=javascript]
rs.initiate({
    _id: 'rs0',
    members: [
        { _id: 0, host: 'central-mongodb:27017', priority: 10 },
        { _id: 1, host: 'secondary-mongodb-1:27017', priority: 5 },
        { _id: 2, host: 'secondary-mongodb-2:27017', priority: 1 }
    ]
})
\end{lstlisting}
\textit{(Ruta: /setup-replica.sh - Configuración de ReplicaSet con esquema de prioridad)}

\paragraph{Proceso de Elección}
El proceso de elección de nodo primario se rige por las siguientes reglas:

\begin{enumerate}
    \item Los nodos participan en una elección cuando:
        \begin{itemize}
            \item El nodo primario actual se vuelve inaccesible
            \item Se produce un reinicio planificado del nodo primario
            \item Se detecta una partición de red
        \end{itemize}
    \item El nodo con la prioridad más alta disponible gana la elección
    \item Se requiere una mayoría de nodos (al menos 2 de 3) para completar una elección
\end{enumerate}

\subsubsection{Implementación de Seguridad}
La seguridad del ReplicaSet se implementa mediante múltiples capas de protección:

\paragraph{Autenticación por Keyfile}
Los nodos del ReplicaSet utilizan un archivo de clave compartida para la autenticación interna:

\begin{lstlisting}[language=yaml]
command: mongod --replSet rs0 --keyFile /etc/mongo-keyfile --bind_ip_all --auth --port 27017
volumes:
    - ./mongo-keyfile:/etc/mongo-keyfile:ro
\end{lstlisting}
\textit{(Ruta: /docker-compose.yml - Configuración de seguridad con keyfile)}

El keyfile se monta como volumen de solo lectura en cada contenedor, garantizando que solo los nodos autorizados puedan unirse al ReplicaSet.

\paragraph{Control de Acceso Basado en Roles (RBAC)}
Se implementa un esquema de RBAC que define roles específicos para diferentes tipos de operaciones:

\begin{lstlisting}[language=javascript]
db.createUser({
    user: 'admin',
    pwd: 'password',
    roles: [
        { role: 'readWrite', db: 'social_media_db' },
        { role: 'dbAdmin', db: 'social_media_db' },
        { role: 'userAdmin', db: 'social_media_db' }
    ]
});
\end{lstlisting}
\textit{(Ruta: /setup-replica.sh - Configuración de control de acceso basado en roles)}

\paragraph{Aislamiento de Red}
Los contenedores operan en una red dedicada (ddbp\_mongo-network) que proporciona:

\begin{itemize}
    \item Aislamiento del tráfico de base de datos
    \item DNS interno para resolución de nombres entre contenedores
    \item Control de acceso a nivel de red
\end{itemize}

\begin{lstlisting}[language=yaml]
networks:
    ddbp_mongo-network:
        name: ddbp_mongo-network
\end{lstlisting}
\textit{(Ruta: /docker-compose.yml - Definición de red dedicada para MongoDB)}

\subsubsection{Proceso de Inicialización Automatizada}
La inicialización del ReplicaSet se automatiza mediante un contenedor dedicado (mongo-setup) que ejecuta un script de configuración:

\begin{enumerate}
    \item \textbf{Verificación de Disponibilidad}: El script verifica que todos los nodos estén operativos antes de iniciar la configuración.
    \item \textbf{Inicialización del ReplicaSet}: Se configura la topología del ReplicaSet con las prioridades correspondientes.
    \item \textbf{Creación de Usuarios y Colecciones}: Se establecen los usuarios administrativos y las colecciones iniciales.
    \item \textbf{Verificación de Estado}: Se confirma el correcto funcionamiento del ReplicaSet mediante pruebas de estado.
\end{enumerate}

El script de configuración incluye mecanismos de reintento y validación:

\begin{lstlisting}[language=bash]
#!/bin/bash
check_mongodb_ready() {
    local host=$1
    echo "Checking if $host is ready..."
    for i in {1..30}; do
        if mongosh --host $host --port 27017 -u admin -p password \
            --authenticationDatabase admin --eval "db.adminCommand('ping')" &>/dev/null; then
            echo "$host is ready!"
            return 0
        fi
        echo "Waiting for $host to be ready (attempt $i/30)..."
        sleep 2
    done
    return 1
}
\end{lstlisting}
\textit{(Ruta: /setup-replica.sh - Script de verificación de disponibilidad de nodos MongoDB)}

Esta configuración automatizada garantiza una inicialización consistente y segura del clúster de MongoDB, estableciendo la base para operaciones distribuidas confiables.

\subsubsection{Connection Pooling y Resiliencia}
La aplicación implementa una configuración robusta para la conexión a MongoDB que garantiza alta disponibilidad, rendimiento óptimo y recuperación ante fallos:

\paragraph{Pool de Conexiones}
\begin{lstlisting}[language=rust]
// Configurar connection pooling
client_options.max_pool_size = Some(20);
client_options.min_pool_size = Some(5);
client_options.max_idle_time = Some(Duration::from_secs(60));
\end{lstlisting}
\textit{(Ruta: /rust-app/src/config.rs - Configuración del pool de conexiones a MongoDB)}

\begin{itemize}
    \item \textbf{Max Pool Size}: Limita el número máximo de conexiones concurrentes a 20 para prevenir saturación.
    \item \textbf{Min Pool Size}: Mantiene al menos 5 conexiones disponibles para minimizar latencia en solicitudes.
    \item \textbf{Max Idle Time}: Cierra conexiones inactivas después de 60 segundos para optimizar recursos.
\end{itemize}

\paragraph{Configuración de Timeouts}
\begin{lstlisting}[language=rust]
// Configurar timeouts
client_options.connect_timeout = Some(Duration::from_secs(10));
client_options.server_selection_timeout = Some(Duration::from_secs(15));
\end{lstlisting}
\textit{(Ruta: /rust-app/src/config.rs - Configuración de timeouts para conexiones a MongoDB)}

\begin{itemize}
    \item \textbf{Connect Timeout}: 10 segundos para establecer la conexión inicial.
    \item \textbf{Server Selection Timeout}: 15 segundos para seleccionar un servidor disponible.
\end{itemize}

\paragraph{Configuración de Consistencia y Durabilidad}
\begin{lstlisting}[language=rust]
// Configurar read/write concerns para mejor fiabilidad
client_options.read_concern = Some(ReadConcern::majority());
client_options.write_concern = Some(
    WriteConcern::builder()
        .w(mongodb::options::Acknowledgment::Majority)
        .build(),
);
\end{lstlisting}
\textit{(Ruta: /rust-app/src/config.rs - Configuración de consistencia y durabilidad para operaciones de MongoDB)}

\begin{itemize}
    \item \textbf{Read Concern}: Configurado como \texttt{majority} para garantizar que las lecturas devuelvan datos confirmados por la mayoría de nodos.
    \item \textbf{Write Concern}: Configurado para requerir confirmación de la mayoría de nodos antes de considerar una escritura como exitosa.
\end{itemize}

\paragraph{Balanceo de Carga y Preferencia de Lectura}
\begin{lstlisting}[language=rust]
// Establecer preferencia de lectura a SecondaryPreferred
let options = ReadPreferenceOptions::default();
client_options.selection_criteria = Some(
    ReadPreference::SecondaryPreferred {
        options: Some(options),
    }
    .into(),
);
\end{lstlisting}
\textit{(Ruta: /rust-app/src/config.rs - Configuración de preferencia de lectura para balanceo de carga)}

\begin{itemize}
    \item \textbf{Read Preference}: Configurado como \texttt{SecondaryPreferred} para dirigir operaciones de lectura a nodos secundarios cuando estén disponibles, descargando así al primario.
\end{itemize}

\paragraph{Políticas de Reintento y Monitoreo}
\begin{lstlisting}[language=rust]
// Configurar comportamiento de reintentos
client_options.retry_reads = Some(true);
client_options.retry_writes = Some(true);

// Configurar heartbeat para detectar problemas rápidamente
client_options.heartbeat_freq = Some(Duration::from_secs(15));
\end{lstlisting}
\textit{(Ruta: /rust-app/src/config.rs - Configuración de políticas de reintento y monitoreo)}

\begin{itemize}
    \item \textbf{Retry Reads/Writes}: Habilitados para reintentar automáticamente operaciones fallidas.
    \item \textbf{Heartbeat}: Configurado para verificar el estado de los servidores cada 15 segundos.
\end{itemize}

Esta configuración integral garantiza que la aplicación pueda manejar eficientemente situaciones como:
\begin{itemize}
    \item Fallos temporales de red
    \item Caída de nodos individuales
    \item Elecciones de nuevo primario
    \item Picos de carga
    \item Desconexiones transitorias
\end{itemize}

\subsection{Diagrama de Comunicación Rust}
\begin{figure}[H]
    \centering
    \begin{tikzpicture}[node distance=2cm]
        % Application Components
        \node[rectangle,draw,minimum width=3cm,minimum height=1.5cm] (app) {Rust Application};
        \node[rectangle,draw,minimum width=3cm,minimum height=1.5cm,below=2cm of app] (primary) {MongoDB Primary};
        \node[rectangle,draw,minimum width=3cm,minimum height=1.5cm,below left=2cm of primary] (secondary1) {MongoDB Secondary 1};
        \node[rectangle,draw,minimum width=3cm,minimum height=1.5cm,below right=2cm of primary] (secondary2) {MongoDB Secondary 2};
        
        % Connections with labels
        \draw[->,thick] (app) -- (primary) node[midway,right] {Writes};
        \draw[->,thick,dashed] (app) -- (secondary1) node[midway,left] {Reads};
        \draw[->,thick,dashed] (app) -- (secondary2) node[midway,right] {Reads};
        
        % Replication arrows
        \draw[->,thick] (primary) -- (secondary1) node[midway,left] {Sync};
        \draw[->,thick] (primary) -- (secondary2) node[midway,right] {Sync};
    \end{tikzpicture}
    \caption{Diagrama de comunicación de la aplicación Rust con MongoDB ReplicaSet}
    \label{fig:rust-communication}
\end{figure}

\section{Casos de Uso}
El sistema de Red Social Distribuida basado en MongoDB implementa diversos casos de uso fundamentales que definen la interacción entre usuarios y el sistema. A continuación, se detallan los principales casos de uso organizados por categorías funcionales.

\subsection{Autenticación y Gestión de Perfiles}
\begin{figure}[H]
    \centering
    \begin{tikzpicture}[node distance=2cm]
        % Actor
        \node[draw, circle, minimum size=1.5cm] (user) {Usuario};
        
        % Use Cases
        \node[draw, ellipse, right=2cm of user, minimum width=3cm, minimum height=1cm] (register) {Registro};
        \node[draw, ellipse, right=2cm of user, below=0.8cm of register, minimum width=3cm, minimum height=1cm] (login) {Iniciar Sesión};
        \node[draw, ellipse, right=2cm of user, below=0.8cm of login, minimum width=3cm, minimum height=1cm] (profile) {Editar Perfil};
        \node[draw, ellipse, right=2cm of user, below=0.8cm of profile, minimum width=3cm, minimum height=1cm] (password) {Cambiar Contraseña};
        
        % Connections
        \draw[-] (user) -- (register);
        \draw[-] (user) -- (login);
        \draw[-] (user) -- (profile);
        \draw[-] (user) -- (password);
    \end{tikzpicture}
    \caption{Casos de uso: Autenticación y Gestión de Perfiles}
    \label{fig:auth-use-cases}
\end{figure}

\paragraph{Descripción de Casos de Uso}
\begin{itemize}
    \item \textbf{Registro de Usuario}:
    \begin{itemize}
        \item \textbf{Actor}: Usuario no registrado
        \item \textbf{Descripción}: El usuario proporciona nombre de usuario, correo electrónico y contraseña para crear una cuenta
        \item \textbf{Flujo Principal}:
        \begin{enumerate}
            \item El usuario ingresa datos personales
            \item El sistema valida la unicidad del nombre de usuario y correo
            \item El sistema almacena credenciales con contraseña hasheada
            \item Se crea perfil de usuario vacío
        \end{enumerate}
        \item \textbf{Postcondición}: Cuenta creada y usuario autenticado
    \end{itemize}
    
    \item \textbf{Inicio de Sesión}:
    \begin{itemize}
        \item \textbf{Actor}: Usuario registrado
        \item \textbf{Descripción}: El usuario introduce credenciales para acceder
        \item \textbf{Flujo Principal}:
        \begin{enumerate}
            \item Usuario ingresa nombre de usuario/email y contraseña
            \item Sistema verifica credenciales contra la base de datos
            \item Sistema genera token de autenticación
        \end{enumerate}
        \item \textbf{Postcondición}: Usuario autenticado y sesión iniciada
    \end{itemize}
    
    \item \textbf{Edición de Perfil}:
    \begin{itemize}
        \item \textbf{Actor}: Usuario autenticado
        \item \textbf{Descripción}: Modificación de información de perfil
        \item \textbf{Flujo Principal}:
        \begin{enumerate}
            \item Usuario accede a sección de perfil
            \item Modifica campos (bio, avatar, nombre completo)
            \item Sistema valida y almacena cambios
        \end{enumerate}
        \item \textbf{Postcondición}: Perfil actualizado en la base de datos
    \end{itemize}
\end{itemize}

\subsection{Gestión de Publicaciones}
\begin{figure}[H]
    \centering
    \begin{tikzpicture}[node distance=2cm]
        % Actor
        \node[draw, circle, minimum size=1.5cm] (user) {Usuario};
        
        % Use Cases
        \node[draw, ellipse, right=2cm of user, minimum width=3cm, minimum height=1cm] (create) {Crear Post};
        \node[draw, ellipse, right=2cm of user, below=0.8cm of create, minimum width=3cm, minimum height=1cm] (view) {Ver Post};
        \node[draw, ellipse, right=2cm of user, below=0.8cm of view, minimum width=3cm, minimum height=1cm] (delete) {Eliminar Post};
        \node[draw, ellipse, right=2cm of user, below=0.8cm of delete, minimum width=3cm, minimum height=1cm] (edit) {Editar Post};
        
        % Connections
        \draw[-] (user) -- (create);
        \draw[-] (user) -- (view);
        \draw[-] (user) -- (delete);
        \draw[-] (user) -- (edit);
    \end{tikzpicture}
    \caption{Casos de uso: Gestión de Publicaciones}
    \label{fig:post-use-cases}
\end{figure}

\paragraph{Descripción de Casos de Uso}
\begin{itemize}
    \item \textbf{Crear Publicación}:
    \begin{itemize}
        \item \textbf{Actor}: Usuario autenticado
        \item \textbf{Descripción}: Creación de nueva publicación con contenido
        \item \textbf{Flujo Principal}:
        \begin{enumerate}
            \item Usuario redacta contenido textual
            \item Opcionalmente adjunta medios (imágenes, videos)
            \item Opcionalmente agrega etiquetas y ubicación
            \item Sistema almacena publicación y actualiza contadores
        \end{enumerate}
        \item \textbf{Postcondición}: Publicación visible para seguidores
    \end{itemize}
    
    \item \textbf{Ver Publicación}:
    \begin{itemize}
        \item \textbf{Actor}: Usuario (autenticado o no)
        \item \textbf{Descripción}: Visualización de una publicación específica
        \item \textbf{Flujo Principal}:
        \begin{enumerate}
            \item Usuario accede a publicación mediante enlace o feed
            \item Sistema recupera publicación con comentarios asociados
            \item Se muestra contenido completo e interacciones
        \end{enumerate}
        \item \textbf{Postcondición}: Publicación visualizada con estadísticas actualizadas
    \end{itemize}
    
    \item \textbf{Eliminar Publicación}:
    \begin{itemize}
        \item \textbf{Actor}: Usuario autenticado (propietario)
        \item \textbf{Descripción}: Eliminación permanente de una publicación
        \item \textbf{Flujo Principal}:
        \begin{enumerate}
            \item Usuario selecciona opción eliminar en su publicación
            \item Sistema solicita confirmación
            \item Sistema elimina publicación y actualiza contadores
        \end{enumerate}
        \item \textbf{Postcondición}: Publicación eliminada y referencias actualizadas
    \end{itemize}
\end{itemize}

\subsection{Interacciones Sociales}
\begin{figure}[H]
    \centering
    \begin{tikzpicture}[node distance=2cm]
        % Actor
        \node[draw, circle, minimum size=1.5cm] (user) {Usuario};
        
        % Use Cases
        \node[draw, ellipse, right=2cm of user, minimum width=3cm, minimum height=1cm] (follow) {Seguir Usuario};
        \node[draw, ellipse, right=2cm of user, below=0.8cm of follow, minimum width=3cm, minimum height=1cm] (like) {Dar Like};
        \node[draw, ellipse, right=2cm of user, below=0.8cm of like, minimum width=3cm, minimum height=1cm] (comment) {Comentar};
        \node[draw, ellipse, right=2cm of user, below=0.8cm of comment, minimum width=3cm, minimum height=1cm] (share) {Compartir Post};
        
        % Connections
        \draw[-] (user) -- (follow);
        \draw[-] (user) -- (like);
        \draw[-] (user) -- (comment);
        \draw[-] (user) -- (share);
    \end{tikzpicture}
    \caption{Casos de uso: Interacciones Sociales}
    \label{fig:social-use-cases}
\end{figure}

\paragraph{Descripción de Casos de Uso}
\begin{itemize}
    \item \textbf{Seguir Usuario}:
    \begin{itemize}
        \item \textbf{Actor}: Usuario autenticado
        \item \textbf{Descripción}: Establecer relación de seguimiento
        \item \textbf{Flujo Principal}:
        \begin{enumerate}
            \item Usuario visita perfil de otro usuario
            \item Usuario selecciona opción "Seguir"
            \item Sistema registra relación en colección de follows
            \item Contadores de ambos usuarios son actualizados
        \end{enumerate}
        \item \textbf{Postcondición}: Relación establecida y contadores actualizados
    \end{itemize}
    
    \item \textbf{Dar Like}:
    \begin{itemize}
        \item \textbf{Actor}: Usuario autenticado
        \item \textbf{Descripción}: Interacción positiva con publicación
        \item \textbf{Flujo Principal}:
        \begin{enumerate}
            \item Usuario visualiza publicación
            \item Usuario selecciona opción "Me gusta"
            \item Sistema registra interacción y actualiza contador
        \end{enumerate}
        \item \textbf{Postcondición}: Like registrado y contador actualizado
    \end{itemize}
    
    \item \textbf{Comentar}:
    \begin{itemize}
        \item \textbf{Actor}: Usuario autenticado
        \item \textbf{Descripción}: Añadir comentario a publicación
        \item \textbf{Flujo Principal}:
        \begin{enumerate}
            \item Usuario visualiza publicación
            \item Usuario redacta comentario
            \item Sistema almacena comentario asociado a la publicación
            \item Contador de comentarios se actualiza
        \end{enumerate}
        \item \textbf{Postcondición}: Comentario visible y contador actualizado
    \end{itemize}
\end{itemize}

\subsection{Gestión de Feed}
\begin{figure}[H]
    \centering
    \begin{tikzpicture}[node distance=2cm]
        % Actor
        \node[draw, circle, minimum size=1.5cm] (user) {Usuario};
        
        % Use Cases
        \node[draw, ellipse, right=2cm of user, minimum width=3cm, minimum height=1cm] (timeline) {Ver Timeline};
        \node[draw, ellipse, right=2cm of user, below=0.8cm of timeline, minimum width=3cm, minimum height=1cm] (discover) {Descubrir Contenido};
        \node[draw, ellipse, right=2cm of user, below=0.8cm of discover, minimum width=3cm, minimum height=1cm] (search) {Buscar Contenido};
        
        % Connections
        \draw[-] (user) -- (timeline);
        \draw[-] (user) -- (discover);
        \draw[-] (user) -- (search);
    \end{tikzpicture}
    \caption{Casos de uso: Gestión de Feed}
    \label{fig:feed-use-cases}
\end{figure}

\paragraph{Descripción de Casos de Uso}
\begin{itemize}
    \item \textbf{Ver Timeline}:
    \begin{itemize}
        \item \textbf{Actor}: Usuario autenticado
        \item \textbf{Descripción}: Visualización de publicaciones recientes de usuarios seguidos
        \item \textbf{Flujo Principal}:
        \begin{enumerate}
            \item Usuario accede a la página principal
            \item Sistema recupera publicaciones de usuarios seguidos
            \item Sistema ordena publicaciones cronológicamente
            \item Se muestra feed personalizado
        \end{enumerate}
        \item \textbf{Postcondición}: Feed actualizado visualizado
    \end{itemize}
    
    \item \textbf{Descubrir Contenido}:
    \begin{itemize}
        \item \textbf{Actor}: Usuario autenticado
        \item \textbf{Descripción}: Exploración de contenido popular o recomendado
        \item \textbf{Flujo Principal}:
        \begin{enumerate}
            \item Usuario accede a sección "Descubrir"
            \item Sistema selecciona contenido popular o relevante
            \item Se presenta contenido de usuarios no seguidos
        \end{enumerate}
        \item \textbf{Postcondición}: Usuario expuesto a nuevo contenido
    \end{itemize}
    
    \item \textbf{Buscar Contenido}:
    \begin{itemize}
        \item \textbf{Actor}: Usuario
        \item \textbf{Descripción}: Búsqueda por términos específicos
        \item \textbf{Flujo Principal}:
        \begin{enumerate}
            \item Usuario ingresa términos de búsqueda
            \item Sistema busca en publicaciones, perfiles y etiquetas
            \item Resultados mostrados por relevancia
        \end{enumerate}
        \item \textbf{Postcondición}: Resultados relevantes presentados al usuario
    \end{itemize}
\end{itemize}

\section{Stack Tecnológico}
El sistema de Red Social Distribuida se construye sobre una arquitectura moderna compuesta por múltiples tecnologías integradas, cada una seleccionada por sus fortalezas específicas que contribuyen al rendimiento, escalabilidad y mantenibilidad del sistema.

\begin{figure}[H]
    \centering
    \begin{tikzpicture}[node distance=1.5cm]
        % Define the layers
        \node[rectangle, minimum width=12cm, minimum height=1.5cm, fill=blue!20, draw] (ui) {Frontend (?)};
        
        \node[rectangle, minimum width=12cm, minimum height=1.5cm, fill=green!20, draw, below=0.5cm of ui] (api) {API Gateway (Actix Web)};
        
        \node[rectangle, minimum width=12cm, minimum height=1.5cm, fill=orange!20, draw, below=0.5cm of api] (app) {Aplicación Backend (Rust)};
        
        \node[rectangle, minimum width=12cm, minimum height=1.5cm, fill=red!20, draw, below=0.5cm of app] (db) {Base de Datos (MongoDB ReplicaSet)};
        
        \node[rectangle, minimum width=12cm, minimum height=1.5cm, fill=gray!20, draw, below=0.5cm of db] (infra) {Infraestructura (Podman/Docker Compose)};
        
        % Connect the layers
        \draw[-stealth, thick] (ui) -- (api);
        \draw[-stealth, thick] (api) -- (app);
        \draw[-stealth, thick] (app) -- (db);
        \draw[-stealth, thick] (db) -- (infra);
        
        % Add labels to the sides
        \node[right=0.5cm of ui, align=left] {Interfaz de usuario};
        \node[right=0.5cm of api, align=left] {Ruteo y manejo de peticiones HTTP};
        \node[right=0.5cm of app, align=left] {Lógica de negocio y servicios};
        \node[right=0.5cm of db, align=left] {Persistencia de datos distribuida};
        \node[right=0.5cm of infra, align=left] {Orquestación de contenedores};
    \end{tikzpicture}
    \caption{Arquitectura de capas del stack tecnológico}
    \label{fig:tech-stack-layers}
\end{figure}

\subsection{Componentes del Stack Tecnológico}

\subsubsection{Lenguajes y Frameworks}

\begin{itemize}
    \item \textbf{Rust}: Lenguaje de programación de alto rendimiento y seguridad de memoria utilizado para el desarrollo del backend.
    \item \textbf{Actix Web}: Framework asíncrono para Rust que permite construir aplicaciones web rápidas y eficientes.
    \item \textbf{MongoDB}: Base de datos NoSQL orientada a documentos, utilizada para almacenar datos de manera flexible y escalable.
    \item \textbf{Podman}: Plataforma de contenedorización que permite empaquetar aplicaciones y sus dependencias en contenedores ligeros.
    \item \textbf{Podman Compose}: Herramienta para definir y ejecutar aplicaciones multicontenedor, facilitando la orquestación de servicios.
    \item \textbf{Tokio}: Runtime asíncrono para Rust que permite la ejecución eficiente de operaciones de entrada/salida.
    \item \textbf{Serde}: Biblioteca de serialización/deserialización para Rust, utilizada para convertir estructuras de datos a y desde JSON.
    \item \textbf{Yaak}: Herramiento para testear las APIs de Actix Web, permitiendo visualizar y manejar las peticiones y respuestas de manera sencilla.

\end{itemize}

\subsection{Estructura de la Aplicación Rust}
\subsubsection{API Endpoints y Funcionalidades}
\subsubsection{Manejo de Errores y Responses}
\subsubsection{Modelos de Datos}

% 3. Modelos de Datos
\section{Modelos de Datos}
\subsection{Diseño de Colecciones}
El sistema implementa un modelo de datos distribuido optimizado para operaciones sociales y escalabilidad horizontal. Las colecciones están diseñadas para maximizar la eficiencia en un entorno distribuido mientras mantienen la consistencia de los datos.

\subsubsection{Colección de Usuarios}
\begin{lstlisting}[language=json]
{
    "_id": ObjectId(),
    "username": "string",         // Único, indexado
    "email": "string",           // Único, indexado
    "password_hash": "string",
    "profile": {
        "full_name": "string",
        "bio": "string",
        "avatar_url": "string",
        "created_at": ISODate(),
        "last_login": ISODate()
    },
    "followers_count": NumberInt, // Contador denormalizado
    "following_count": NumberInt, // Contador denormalizado
    "posts_count": NumberInt      // Contador denormalizado
}
\end{lstlisting}

\paragraph{Índices y Optimización}
\begin{itemize}
    \item Índice único en \texttt{username}: \texttt{\{username: 1\}}
    \item Índice único en \texttt{email}: \texttt{\{email: 1\}}
    \item Índice compuesto para búsquedas: \texttt{\{username: 1, email: 1\}}
\end{itemize}

\subsubsection{Colección de Posts}
\begin{lstlisting}[language=json]
{
    "_id": ObjectId(),
    "user_id": ObjectId(),      // Referencia al autor
    "content": "string",
    "media_urls": ["string"],   // URLs de contenido multimedia
    "created_at": ISODate(),
    "likes_count": NumberInt,   // Contador denormalizado
    "comments_count": NumberInt, // Contador denormalizado
    "tags": ["string"],
    "location": {
        "type": "Point",
        "coordinates": [NumberDouble, NumberDouble]
    }
}
\end{lstlisting}

\paragraph{Índices y Optimización}
\begin{itemize}
    \item Índice en \texttt{user\_id}: \texttt{\{user\_id: 1\}}
    \item Índice en \texttt{created\_at}: \texttt{\{created\_at: -1\}}
    \item Índice geoespacial: \texttt{\{location: "2dsphere"\}}
    \item Índice de texto en contenido: \texttt{\{content: "text", tags: "text"\}}
\end{itemize}

\subsubsection{Colección de Comentarios}
\begin{lstlisting}[language=json]
{
    "_id": ObjectId(),
    "post_id": ObjectId(),        // Referencia al post
    "user_id": ObjectId(),        // Referencia al autor
    "content": "string",
    "created_at": ISODate(),
    "likes_count": NumberInt,     // Contador denormalizado
    "parent_comment_id": ObjectId() // Para comentarios anidados
}
\end{lstlisting}

\paragraph{Índices y Optimización}
\begin{itemize}
    \item Índice compuesto: \texttt{\{post\_id: 1, created\_at: -1\}}
    \item Índice en \texttt{user\_id}: \texttt{\{user\_id: 1\}}
    \item Índice en \texttt{parent\_comment\_id}: \texttt{\{parent\_comment\_id: 1\}}
\end{itemize}

\subsubsection{Colección de Likes}
\begin{lstlisting}[language=json]
{
    "_id": ObjectId(),
    "user_id": ObjectId(),
    "target_id": ObjectId(),  // ID del post o comentario
    "target_type": "string",  // "post" o "comment"
    "created_at": ISODate()
}
\end{lstlisting}

\subsubsection{Colección de Follows}
\begin{lstlisting}[language=json]
{
    "_id": ObjectId(),
    "follower_id": ObjectId(),
    "following_id": ObjectId(),
    "created_at": ISODate()
}
\end{lstlisting}

\subsection{Estrategias de Modelado de Datos}

\subsubsection{Denormalización Estratégica}
El sistema implementa denormalización selectiva para optimizar las operaciones de lectura más comunes:

\begin{itemize}
    \item \textbf{Contadores Precalculados}: Mantenimiento de contadores en documentos de usuario para followers, following y posts.
    \item \textbf{Feed Desnormalizado}: Almacenamiento de información frecuentemente accedida en la colección de feed.
    \item \textbf{Información de Perfil Embebida}: Datos de perfil almacenados directamente en el documento de usuario.
\end{itemize}

\subsubsection{Gestión de Relaciones}
Las relaciones entre entidades se manejan mediante referencias, optimizando para:

\begin{itemize}
    \item \textbf{Escalabilidad}: Las referencias permiten que los documentos crezcan independientemente.
    \item \textbf{Flexibilidad}: Facilita la modificación de esquemas y la evolución de la aplicación.
    \item \textbf{Consultas Eficientes}: Permite agregaciones y búsquedas optimizadas.
\end{itemize}

\subsection{Distribución de Datos}
La distribución de datos en el ReplicaSet se optimiza para garantizar:

\paragraph{Consistencia}
\begin{itemize}
    \item Escrituras confirmadas por mayoría de nodos
    \item Lecturas consistentes desde secundarios
    \item Manejo de conflictos mediante timestamps
\end{itemize}

\paragraph{Disponibilidad}
\begin{itemize}
    \item Replicación automática entre nodos
    \item Failover transparente para aplicaciones cliente
    \item Recuperación automática de nodos caídos
\end{itemize}

\paragraph{Particionamiento}
Aunque el sistema actual utiliza un ReplicaSet sin sharding, está diseñado para facilitar la futura implementación de sharding horizontal:

\begin{itemize}
    \item Claves de documento distribuibles (\texttt{\_id})
    \item Índices preparados para sharding
    \item Modelo de datos compatible con distribución
\end{itemize}

\subsection{Estrategias de Indexación}
La estrategia de indexación está diseñada para optimizar los patrones de acceso más comunes:

\begin{itemize}
    \item \textbf{Índices Únicos}: Garantizan la unicidad de usernames y emails
    \item \textbf{Índices Compuestos}: Optimizan consultas multi-campo frecuentes
    \item \textbf{Índices Geoespaciales}: Facilitan búsquedas basadas en ubicación
    \item \textbf{Índices de Texto}: Permiten búsquedas eficientes en contenido
\end{itemize}

Esta estructura de datos distribuida proporciona una base sólida para las operaciones de la red social, permitiendo:

\begin{itemize}
    \item Escalabilidad horizontal futura
    \item Alta disponibilidad de datos
    \item Rendimiento optimizado para operaciones frecuentes
    \item Flexibilidad para evolución del sistema
\end{itemize}

% 4. Endpoints de la API
\section{Endpoints de la API}
\subsection{Operaciones de Usuario}
La API expone endpoints para gestionar operaciones relacionadas con usuarios:

\subsubsection{Crear Usuario}
\begin{lstlisting}[language=rust]
POST /create_user
Content-Type: application/json

{
    "username": "string",
    "email": "string",
    "password": "string",
    "profile": {
        "full_name": "string",
        "bio": "string",
        "avatar_url": "string"
    }
}
\end{lstlisting}

Este endpoint:
\begin{itemize}
    \item Valida datos de entrada
    \item Verifica unicidad de username y email
    \item Hashea la contraseña de forma segura
    \item Crea el documento de usuario en MongoDB
    \item Retorna el ID del usuario creado
\end{itemize}

\subsubsection{Perfiles de Usuario}
\begin{lstlisting}[language=rust]
GET /user/{user_id}

Response:
{
    "user_id": "ObjectId",
    "username": "string",
    "profile": {
        "full_name": "string",
        "bio": "string",
        "avatar_url": "string",
        "created_at": "ISODate",
        "last_login": "ISODate"
    },
    "stats": {
        "followers_count": number,
        "following_count": number,
        "posts_count": number
    }
}
\end{lstlisting}

Funcionalidades:
\begin{itemize}
    \item Recuperación eficiente de perfiles
    \item Proyección de campos según permisos
    \item Caché de perfiles frecuentes
    \item Estadísticas en tiempo real
\end{itemize}

\subsection{Interacciones Sociales}
La API proporciona endpoints para gestionar las interacciones entre usuarios:

\subsubsection{Crear Posts}
\begin{lstlisting}[language=rust]
POST /create_post
Content-Type: application/json

{
    "user_id": "ObjectId",
    "content": "string",
    "media_urls": ["string"],
    "tags": ["string"],
    "location": {
        "type": "Point",
        "coordinates": [double, double]
    }
}
\end{lstlisting}

Características del endpoint:
\begin{itemize}
    \item Validación de usuario existente
    \item Procesamiento de contenido multimedia
    \item Actualización atómica de contadores
    \item Indexación de contenido para búsqueda
\end{itemize}

\subsubsection{Funcionalidad de Comentarios}
\begin{lstlisting}[language=rust]
POST /create_comment
Content-Type: application/json

{
    "post_id": "ObjectId",
    "user_id": "ObjectId",
    "content": "string",
    "parent_comment_id": "ObjectId"  // Opcional
}
\end{lstlisting}

El endpoint maneja:
\begin{itemize}
    \item Comentarios en posts
    \item Respuestas a comentarios existentes
    \item Actualización de contadores de comentarios
    \item Notificaciones a usuarios relevantes
\end{itemize}

\subsubsection{Relaciones de Follow}
\begin{lstlisting}[language=rust]
POST /follow_user
Content-Type: application/json

{
    "follower_id": "ObjectId",
    "following_id": "ObjectId"
}
\end{lstlisting}

Implementación:
\begin{itemize}
    \item Verificación de usuarios existentes
    \item Validación de relación no duplicada
    \item Actualización atómica de contadores
    \item Generación de eventos de notificación
\end{itemize}

\subsection{Operaciones del Sistema}
La API incluye endpoints para monitoreo y mantenimiento del sistema:

\subsubsection{Implementación del Health Check}
\begin{lstlisting}[language=rust]
GET /health

Response:
{
    "status": "Healthy",
    "mongo_status": {
        "primary": "connected",
        "secondaries": ["connected", "connected"],
        "replication_lag": [0, 1]
    }
}
\end{lstlisting}

El health check verifica:
\begin{itemize}
    \item Conectividad con MongoDB
    \item Estado del ReplicaSet
    \item Lag de replicación
    \item Estado del pool de conexiones
\end{itemize}

\subsubsection{Población de Base de Datos para Testing}
\begin{lstlisting}[language=rust]
POST /test/populate
Content-Type: application/json

{
    "users_count": number,
    "posts_per_user": number,
    "comments_per_post": number
}
\end{lstlisting}

Funcionalidades:
\begin{itemize}
    \item Generación de datos de prueba realistas
    \item Creación de relaciones entre entidades
    \item Distribución estadística de interacciones
    \item Verificación de integridad de datos
\end{itemize}

\subsubsection{Funcionalidad de Limpieza de Base de Datos}
\begin{lstlisting}[language=rust]
POST /test/clean

Response:
{
    "status": "success",
    "collections_cleaned": ["users", "posts", "comments", "likes", "follows"]
}
\end{lstlisting}

El endpoint:
\begin{itemize}
    \item Limpia datos de prueba
    \item Mantiene configuración del sistema
    \item Verifica integridad post-limpieza
    \item Registra operación en logs
\end{itemize}

Todos los endpoints implementan:
\begin{itemize}
    \item Manejo robusto de errores
    \item Validación de entrada
    \item Logging de operaciones
    \item Métricas de rendimiento
    \item Control de acceso basado en roles
\end{itemize}
% 5. Implementación Técnica
\section{Implementación Técnica}
\subsection{Error Handling}
La implementación incluye un sistema robusto de manejo de errores que abarca múltiples niveles de la aplicación:

\begin{lstlisting}[language=rust]
#[derive(Debug)]
pub enum AppError {
    // Errores de Base de Datos
    DatabaseError(mongodb::error::Error),
    ConnectionError(String),
    
    // Errores de Validación
    ValidationError(String),
    DuplicateKeyError(String),
    
    // Errores de Negocio
    NotFoundError(String),
    UnauthorizedError(String),
    
    // Errores del Sistema
    SystemError(String)
}
\end{lstlisting}
\textit{(Ruta: /rust-app/src/error.rs - Enumeración de errores para manejo centralizado de excepciones)}

\paragraph{Estrategias de Manejo}
\begin{itemize}
    \item \textbf{Propagación Controlada}: Los errores se propagan de manera controlada mediante Result
    \item \textbf{Logging Estructurado}: Registro detallado de errores con contexto
    \item \textbf{Recuperación Automática}: Reintentos automáticos para errores transitorios
    \item \textbf{Respuestas HTTP}: Mapeo de errores a códigos HTTP apropiados
\end{itemize}

\subsection{State Management}
La gestión del estado de la aplicación se maneja mediante una estructura AppState que encapsula recursos compartidos:

\begin{lstlisting}[language=rust]
pub struct AppState {
    pub db: Database,
    pub config: AppConfig,
    pub metrics: Arc<Metrics>,
}

impl AppState {
    pub async fn new(mongo_uri: &str, config: AppConfig) -> Result<Self, AppError> {
        let client_options = ClientOptions::parse(mongo_uri).await?;
        let client = Client::with_options(client_options)?;
        let db = client.database("social_media_db");
        
        Ok(Self {
            db,
            config,
            metrics: Arc::new(Metrics::new()),
        })
    }
}
\end{lstlisting}
\textit{(Ruta: /rust-app/src/state.rs - Implementación del estado de la aplicación y gestión de recursos compartidos)}

\paragraph{Características Principales}
\begin{itemize}
    \item \textbf{Estado Compartido}: Acceso thread-safe a recursos compartidos
    \item \textbf{Configuración Centralizada}: Gestión unificada de configuración
    \item \textbf{Métricas en Tiempo Real}: Recolección de métricas de rendimiento
    \item \textbf{Conexión a Base de Datos}: Pool de conexiones administrado
\end{itemize}

\subsection{Conexión a Base de Datos y Resiliencia}
La conexión a MongoDB se implementa con énfasis en la resiliencia y rendimiento:

\begin{lstlisting}[language=rust]
async fn setup_database(config: &Config) -> Result<Database, AppError> {
    let mut client_options = ClientOptions::parse(&config.mongo_uri).await?;
    
    // Configuración de timeouts
    client_options.connect_timeout = Some(Duration::from_secs(10));
    client_options.server_selection_timeout = Some(Duration::from_secs(15));
    
    // Configuración de pool de conexiones
    client_options.max_pool_size = Some(20);
    client_options.min_pool_size = Some(5);
    
    // Configuración de consistencia
    client_options.read_concern = Some(ReadConcern::majority());
    client_options.write_concern = Some(
        WriteConcern::builder()
            .w(mongodb::options::Acknowledgment::Majority)
            .build(),
    );
    
    let client = Client::with_options(client_options)?;
    Ok(client.database("social_media_db"))
}
\end{lstlisting}
\textit{(Ruta: /rust-app/src/main.rs - Configuración y conexión a la base de datos MongoDB)}

\paragraph{Características del Testing}
\begin{itemize}
    \item \textbf{Base de Datos de Prueba}: Instancia dedicada para testing
    \item \textbf{Fixtures Automatizados}: Generación de datos de prueba
    \item \textbf{Limpieza Automática}: Restauración del estado inicial
    \item \textbf{Tests de Integración}: Pruebas end-to-end con base de datos real
\end{itemize}

Esta implementación técnica asegura:
\begin{itemize}
    \item Alta calidad del código
    \item Mantenibilidad a largo plazo
    \item Facilidad de debugging
    \item Confiabilidad del sistema
\end{itemize}
\section{Manejo de Consistencia y Durabilidad}
\subsection{Configuración de Garantías de Consistencia}
La arquitectura implementa un modelo de consistencia fuerte mediante configuraciones específicas que garantizan la durabilidad y consistencia de los datos en el entorno distribuido:

\paragraph{Write Concern Mayoría}
La configuración de write concern asegura que las operaciones de escritura sean confirmadas por la mayoría de los nodos:

\begin{lstlisting}[language=rust]
client_options.write_concern = Some(
    WriteConcern::builder()
        .w(mongodb::options::Acknowledgment::Majority)
        .build(),
);
\end{lstlisting}
\textit{(Ruta: /rust-app/src/config.rs - Configuración de garantías de escritura para durabilidad)}

Esta configuración proporciona:
\begin{itemize}
    \item Garantía de durabilidad ante fallos de nodos
    \item Consistencia entre réplicas para operaciones posteriores
    \item Protección contra pérdida de datos en escenarios de failover
\end{itemize}

\paragraph{Read Concern Mayoría}
Las operaciones de lectura se configuran para garantizar consistencia:

\begin{lstlisting}[language=rust]
client_options.read_concern = Some(ReadConcern::majority());
\end{lstlisting}
\textit{(Ruta: /rust-app/src/config.rs - Configuración de garantías de lectura para consistencia)}

Este nivel de read concern asegura:
\begin{itemize}
    \item Lecturas de datos confirmados por la mayoría de réplicas
    \item Prevención de lecturas sucias (dirty reads)
    \item Consistencia en operaciones distribuidas
\end{itemize}

\section{Configuración de Resiliencia}
\subsection{Gestión de Timeouts}
\paragraph{Timeouts Operacionales}
\begin{lstlisting}[language=rust]
client_options.connect_timeout = Some(Duration::from_secs(10));
client_options.server_selection_timeout = Some(Duration::from_secs(15));
client_options.max_idle_time = Some(Duration::from_secs(60));
\end{lstlisting}
\textit{(Ruta: /rust-app/src/config.rs - Configuración de timeouts operacionales para resiliencia)}

Los timeouts se configuran para:
\begin{itemize}
    \item \textbf{Connect Timeout (10s)}: Límite para conexiones iniciales
    \item \textbf{Server Selection (15s)}: Tiempo máximo para selección de servidor
    \item \textbf{Max Idle Time (60s)}: Duración máxima de conexiones inactivas
\end{itemize}

\subsection{Políticas de Reintento}
\begin{lstlisting}[language=rust]
client_options.retry_reads = Some(true);
client_options.retry_writes = Some(true);
\end{lstlisting}
\textit{(Ruta: /rust-app/src/config.rs - Configuración de políticas de reintento para operaciones)}

Las operaciones elegibles para reintento incluyen:
\begin{itemize}
    \item Errores de red transitorios
    \item Timeouts de operaciones
    \item Errores de escritura no fatales
    \item Reconexiones post-failover
\end{itemize}

\subsection{Monitoreo y Health Checks}
\paragraph{Monitoreo de Base de Datos}
\begin{lstlisting}[language=rust]
client_options.heartbeat_freq = Some(Duration::from_secs(15));
\end{lstlisting}
\textit{(Ruta: /rust-app/src/config.rs - Configuración de frecuencia de heartbeat para monitoreo)}

La implementación incluye:
\begin{itemize}
    \item \textbf{Heartbeat}: Verificación periódica de nodos
    \item \textbf{Métricas de Replicación}: Monitoreo de lag y sincronización
    \item \textbf{Estadísticas Operacionales}: Seguimiento de latencia y throughput
\end{itemize}

\paragraph{Health Checks de Aplicación}
\begin{lstlisting}[language=rust]
async fn health_check_handler(
    app_state: web::Data<AppState>
) -> impl Responder {
    match app_state.db.run_command(doc! {"ping": 1}, None).await {
        Ok(_) => HttpResponse::Ok().json("Healthy"),
        Err(e) => HttpResponse::ServiceUnavailable().json(e.to_string())
    }
}
\end{lstlisting}
\textit{(Ruta: /rust-app/src/handlers.rs - Manejador de verificación de salud del sistema)}

El sistema implementa verificaciones de:
\begin{itemize}
    \item Conectividad a MongoDB
    \item Estado del pool de conexiones
    \item Métricas de rendimiento
    \item Uso de recursos
\end{itemize}

\subsection{Balanceo de Carga}
La aplicación implementa estrategias de balanceo en múltiples niveles:

\paragraph{Nivel de Base de Datos}
\begin{lstlisting}[language=rust]
client_options.selection_criteria = Some(
    ReadPreference::SecondaryPreferred {
        options: Some(ReadPreferenceOptions::default()),
    }
    .into(),
);
\end{lstlisting}
\textit{(Ruta: /rust-app/src/config.rs - Configuración de criterios de selección para balanceo de carga)}

Características principales:
\begin{itemize}
    \item Distribución de lecturas entre secundarios
    \item Escrituras centralizadas en primario
    \item Failover automático
\end{itemize}

\paragraph{Nivel de Aplicación}
\begin{itemize}
    \item Connection pooling optimizado (5-20 conexiones)
    \item Distribución de carga entre instancias
    \item Caché local para datos frecuentes
\end{itemize}

\subsection{Error Handling}
La aplicación implementa un manejo de errores multinivel:

\begin{itemize}
    \item \textbf{Errores de Conexión}: Problemas de red o disponibilidad
    \item \textbf{Errores de Operación}: Fallos en operaciones específicas
    \item \textbf{Errores de Aplicación}: Problemas de lógica de negocio
    \item \textbf{Errores de Sistema}: Fallos de recursos o configuración
\end{itemize}

Estrategias de recuperación:
\begin{itemize}
    \item \textbf{Circuit Breaker}: Prevención de cascada de fallos
    \item \textbf{Backoff Exponencial}: Reintento gradual
    \item \textbf{Fallback}: Rutas alternativas para operaciones críticas
    \item \textbf{Logging}: Registro detallado para análisis
\end{itemize}

\section{Conclusiones}
La implementación del sistema de base de datos distribuida MongoDB para nuestra aplicación de red social demuestra una arquitectura robusta y escalable que cumple con los requisitos de alta disponibilidad, consistencia y rendimiento. Los aspectos más destacados de la implementación incluyen:

\subsection{Arquitectura Distribuida}
\begin{itemize}
    \item Implementación exitosa de un ReplicaSet MongoDB con esquema de prioridad 10-5-1
    \item Sistema de failover automático que garantiza continuidad operativa
    \item Balanceo de carga efectivo entre nodos primario y secundarios
    \item Aislamiento de red y seguridad mediante autenticación por keyfile
\end{itemize}

\subsection{Garantías de Datos}
\begin{itemize}
    \item Consistencia fuerte mediante write concern y read concern mayoritarios
    \item Durabilidad asegurada por replicación síncrona entre nodos
    \item Prevención de pérdida de datos mediante confirmación distribuida
    \item Estrategias de de normalización optimizadas para lecturas frecuentes
\end{itemize}

\subsection{Resiliencia y Monitoreo}
\begin{itemize}
    \item Sistema robusto de manejo de errores y recuperación
    \item Monitoreo continuo mediante heartbeats y health checks
    \item Connection pooling optimizado para rendimiento sostenido
    \item Políticas de reintento configuradas para operaciones críticas
\end{itemize}

\subsection{Escalabilidad}
\begin{itemize}
    \item Diseño preparado para futuro sharding horizontal
    \item Modelo de datos optimizado para distribución
    \item Índices estratégicos para consultas frecuentes
    \item Arquitectura modular que facilita expansión
\end{itemize}

El sistema resultante proporciona una base sólida para las operaciones de la red social, con capacidad de crecimiento y adaptación a futuros requerimientos. La combinación de MongoDB como sistema de base de datos distribuida y Rust como lenguaje de implementación ha demostrado ser efectiva para crear una aplicación robusta y de alto rendimiento.

Las decisiones arquitectónicas tomadas en términos de consistencia, disponibilidad y tolerancia a particiones (CAP) han priorizado la consistencia y disponibilidad, mientras se mantiene la capacidad de escalar horizontalmente cuando sea necesario. La implementación actual sienta las bases para futuras mejoras y optimizaciones según evolucionen los requerimientos del sistema.

\end{document}
